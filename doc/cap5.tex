\paragraph{}Este cap�tulo tem como objetivo apresentar os resultados obtidos ao aplicar o m�todo de estima��o de altura de ondas discutido nos cap�tulos 2 e 3. 

\paragraph{}Os v�deos analisados foram capturados na praia de Itacoatiara, em Niter�i. A c�mera foi posicionada no Quiosque 5, na orla da praia, a uma altura de aproximadamente oito metros do n�vel do mar. Este quiosque foi escolhido para posicionar a c�mera pois uma vis�o clara do mar, sem a obstru��o da vegeta��o local da praia. A c�mera est� posicionada com uma angula��o vertical de 83$^{\circ}$. O campo visual (\textit{field of view}) da c�mera do \textit{smartphone} foi calculado atrav�s da seguinte equa��o \cite{Bourke}: 

\[
	fov = 2 * atan(height/2*focalLength)
\]

\noindent{}onde $height$ � a altura da imagem no sensor da c�mera e $focalLength$ � a dist�ncia focal da c�mera. Estes par�metros foram obtidos no momento da captura atrav�s do pr�prio \textit{smartphone}. Os valores considerados foram:

\begin{center}
    \begin{tabular}{| l | r |}
    \hline
    Par�metros & Valores \\ \hline
    Dist�ncia Focal & 4.15 mm\\ \hline
    Altura do sensor & 3.60 mm\\ \hline
    Campo de vis�o vertical & 46.92$^{\circ}$ \\ 
    \hline
    \end{tabular}
\end{center}

\paragraph{}Foram analisados \todo[inline]{adicionar numero de v�deos} v�deos. Para cada v�deo as ondas foram analisadas automaticamente pelo algoritmo e pelo manualmente pelo autor, a fim de verificar quantas ondas foram ou deixaram de ser identificadas, e verificar a precis�o da identifica��o do ponto m�nimo e m�ximo da onda.