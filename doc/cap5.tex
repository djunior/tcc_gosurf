\paragraph{}Este cap?ulo tem como objetivo apresentar os resultados obtidos ao aplicar o método de estimação de altura de ondas discutido nos capítulos 2 e 3. 

\paragraph{}Os vídeos analisados foram capturados na praia de Itacoatiara, em Niterói. A câmera foi posicionada no Quiosque 5, na orla da praia, a uma altura de aproximadamente oito metros do n?el do mar. Este quiosque foi escolhido para posicionar a câmera pois uma visão clara do mar, sem a obstrução da vegetação local da praia. A câmera está posicionada com uma angulação vertical de 83$^{\circ}$. O campo visual (\textit{field of view}) da câmera do \textit{smartphone} foi calculado através da seguinte equação \cite{Bourke}: 

\[
	fov = 2 * atan(height/2*focalLength)
\]

\noindent{}onde $height$ é a altura da imagem no sensor da câmera e $focalLength$ é a distância focal da câmera. Estes parâmetros foram obtidos no momento da captura através do próprio \textit{smartphone}. Os valores considerados foram:

\begin{center}
    \begin{tabular}{| l | r |}
    \hline
    Parâetros & Valores \\ \hline
    Distância Focal & 4.15 mm\\ \hline
    Altura do sensor & 3.60 mm\\ \hline
    Campo de visão vertical & 46.92$^{\circ}$ \\ 
    \hline
    \end{tabular}
\end{center}

\paragraph{}Foram analisados \todo[inline]{adicionar numero de vídeos} vídeos. Para cada vídeo as ondas foram analisadas automaticamente pelo algoritmo e pelo manualmente pelo autor, a fim de verificar quantas ondas foram ou deixaram de ser identificadas, e verificar a precisão da identificação do ponto mínimo e máximo da onda. Estes são os principais erros observados nos dados analisados. 

\paragraph{}A identificação dos pontos mínimos e máximos está suscetível a interferência externa, como banhistas ou pessoas próximas da zona de arrebentação. A ação do vento \cite{Hwang2016} também pode introduzir erros na identificação do ponto de máximo, causando um \textit{spray} da espuma do mar que pode ser erroneamente detectado como o ponto máximo da onda (Figura \ref{FigMaxError}).

\begin{figure}[h]
	\centering

	\subfloat[]{\includegraphics[width=0.45\textwidth,keepaspectratio]{}}

	\qquad

	\subfloat[]{\includegraphics[width=-0.45\textwidth,keepaspectratio]{}}

	\caption[\small{Comparativo entre onda detectada manualmente e onda detectada automaticamente com efeito de \textit{spray} de espuma.}]{\small{Comparativo entre onda detectada manualmente e onda detectada automaticamente com efeito de \textit{spray} de espuma.}}
	\label{FigMaxError}
\end{figure}